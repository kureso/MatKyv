\documentclass{article}
\usepackage{amsmath}
\usepackage{amssymb}
\title{{\Huge Matematické kyvadlo}}
\author{Ondřej Kureš, Marek Mikloš a Ladislav Trnka\\
Skupina A}
\date{}
\begin{document}
\section{Perturbačná metóda pre nelineárne rovnice}
Na problém sa môžeme pozrieť aj z iného hľaďiska. Môžeme sa pokúsiť rozšíriť nelineárny člen v equation1.1 pomocou Taylorovho rozvoja
\begin{equation}
\label{eq:1}
  \sin \theta  = \sum_{n=1}^{+\infty} (-1)^n \frac{\theta^{2n+1}}{(2k + 1)!} \approx \theta - \frac{\theta^3}{3!} + \frac{\theta^5}{5!} + \cdots
\end{equation}
a následne vyriešiť odpovedajúcu nelineárnu diferenciálnu rovnicu. Touto aproximáciou nám funkciu $\sin \theta $ nahradí polynomiálna funkcia. Výhodou je, že sa nám bude lepšie pracovať s polynomiálnou funkciou.
Ďalej treba podotknúť, že nelineárne členy v rozvoji \eqref{eq:1}, nie sú tak dôležité ako členy nižších stupňov. Otázkou je, ako vybrať parameter, ktorý by bol dostatočne "malý". V našom prípade bude týmto parametrom amplitúda oscilácií. Postup je nasledovný. Ak uvažujeme počiatočné podmienky v rovnici equation3.1 


\begin{subequations}
\begin{equation}
\label{eq:2}
	\frac{d \theta}{dt}\right|_{t=0} = x^2 + x
\end{equation}


\begin{equation}
\label{eq:3}
	\frac{d \theta}{dt}\right|_{t=0} = x^2 + x
\end{equation}
\end{subequations}




\end{document}
